\section{Stripes}

\paragraph*{}
Stripes are flat sections of brightness profile bounded by two step edges of opposite polarity. Such definition indicates that problem of stripe detection heavily depends on already discussed detection of step edges. 

\paragraph*{}
The concept of stripe is important mostly as a clear and succint mean of formulation for a range of visual inspection tasks, wheareas it does not bring any novelties to the signal-processing aspect of the computation. Indeed, algorithms for stripe extraction firstly find the step edges in the profile (using previously described methods) and then process the results combining the extracted edges into stripes. 

\paragraph*{}
Next section summarizes two basic methods of combining the extracted step edges into stripes.

\subsection{Edge Processing}

\subsubsection{All Stripes}

\paragraph*{} As long as our goal is to maximize the number of constructed stripes, the problem can be solved quite efficiently. It can be easily proven that a simple $O(n)$ algorithm that greedily connects each closing edge with the first opening edge between already constructed stripes and the closing edge itself yields optimal results.

\subsubsection{N Stripes}

\paragraph*{}
The task is slightly more complicated if we know the desired number of stripes in advance and aim at maximizing the sum of strengths of step edges constituing the selected stripes. To solve such optimization problem in $O(n^2)$ time we can use a dynamic programming solution.

\paragraph*{}
Let us define a partial solution to the problem as follows:
\begin{description}
	\item \texttt{Best[Prefix][Count]} - sublist of the first \texttt{Prefix} step edges of alternating edge-polarities having \texttt{Count} elements that yields the biggest sum of edge strengths.
\end{description}

\paragraph*{}
Having computed the results for \texttt{Prefix} $=p$ we can compute the results for \texttt{Prefix} $=p+1$ in linear time - for each \texttt{Count} $=c$ we need to consider only two cases, either using the $p+1$-th step edge to extend the optimal result of \texttt{Best[p][count-1]} or not, in which case the result for the subproblem will be equal to \texttt{Best[p][count]}.

\begin{refImpl}
Stripe detection algorithms are implemented in three \studio filters. All of them share common extraction logic, differing only in post-processing method applied to select the final outcome.
\begin{itemize}
	\item \filter{ScanMultipleStripes}{1DEdgeDetection} - maximizes the number of returned stripes.
	\item \filter{ScanExactlyNStripes}{1DEdgeDetection} - constructs the most prominent set of stripes of given cardinality.
	\item \filter{ScanSingleStripe}{1DEdgeDetection} - wrapper over previous filter which selects the single most prominent stripe.
\end{itemize} 
\end{refImpl}
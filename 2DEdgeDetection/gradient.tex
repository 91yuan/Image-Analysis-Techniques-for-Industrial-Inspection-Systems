\section{Image Gradient}

\paragraph*{}
A derivative of a signal is a measure of local change in the signal value, and as such is a natural starting point for detection of step edges, regardless of the dimension of the space in which the signal is defined. A derivative of two-dimensional signal is usually called a \textbf{gradient}. 

In the \refchap{1DEdgeDetection} chapter we have been using partial difference operators precessed by smoothing of the signal to obtain the derivative of a profile and identify its prominent extrema as the edge points. We will translate this approach to two-dimensional space preserving its core idea - calculation of signal derivative using smoothing and partial difference operators.

\subsection{2D Partial Difference}

\paragraph*{}
To calculate the gradient of an image we may apply one of the partial difference operators discussed in the \refchap{1DEdgeDetection} chapter separately in X and Y dimension. \reffig{CentralDifferenceMasks} demonstrates appropriate convolution masks for the central difference operator.

\begin{figure}[h!]
\begin{equation*}
D_{H} = 
	\begin{bmatrix}
	-1 & 0 & 1 \\
	\end{bmatrix}\qquad
D_{V} = 
	\begin{bmatrix}
	-1 \\
	0 \\
	1
	\end{bmatrix}
\end{equation*}
\caption{Horizontal and vertical central difference masks.}
\label{fig:CentralDifferenceMasks}
\end{figure}

\paragraph*{}
In effect we obtain two images (or one two-channel image), representing vertical and horizontal derivatives at each pixel. Combining these two components into vectors representing two-dimensional gradient yields a result demonstrated in \reffig{GradientVectors}, where the resulting gradient vectors were drawn onto the original image.

\oneFigure
{img/2DEdgeDetection/GradientVectors}
{Two-dimensional image derivative computed at each pixel.}
{GradientVectors}
{\basicWidth}

\paragraph*{}
For our needs of edge detection, we are interested mainly in the magnitude of gradient vectors. \reffig{GradientMagnitude} demonstrates a gradient magnitude image, each pixel of which represents the magnitude of the corresponding gradient vector (computed using central difference operators as described above).

\twoFigures
{img/2DEdgeDetection/tooth}
{img/2DEdgeDetection/tooth_pd_amplitude}
{Gradient magnitude image of the example image.}
{GradientMagnitude}
{\basicWidth}

\paragraph*{}
We may notice that image edges are indeed discernible in such image, even though there is also significant amount of high-gradient pixels corresponding to fine features of the saw texture, rather than object edges. Sensitivity to noise was the reason for which we have incorporated Gaussian smoothing operator into our one-dimensional edge detection routines; it is clear that we need similar precautions in two dimensions.

\subsubsection{The need for simple operators}

\paragraph*{}
Even though the Canny's result\cite{Canny86} about the optimality of the derivative of Gaussian operator for edge detection applications holds for 2D, we will commence out description with simpler gradient operators, moving to Canny's method in the next section.

\paragraph*{}
Contrary to the case of one dimension, in two dimensions the amount of data to be processed is usually quite large, which creates a trade-off between accuracy and speed for real-time high-demand inspection systems. It is therefore reasonable to look into simple and fast operators, even if they are less accurate that the one advocated by Canny.

\subsection{Prewitt Operator}

\paragraph*{}
One of the first gradient operators, Prewitt operator, addresses this issue by means similar to the Multiple Sampling technique which we employed to suppress noise in profiles of image brightness at the extraction level. 

\paragraph*{}
Here the central difference operator is applied to three rows/columns in the immediate neighborhood of each pixel, which is equivalent to convolving the central difference operator witch mean average operator running in the opposite direction, as demonstrated in \reffig{PrewittMasks}.

\begin{figure}[h!]
\begin{align*}
P_{H} &= \begin{bmatrix}
	-1 & 0 & 1 
	\end{bmatrix}
	*
	\begin{bmatrix}
	1 \\
	1 \\
	1
	\end{bmatrix}
	= 
	\begin{bmatrix}
	-1 & 0 & 1 \\
	-1 & 0 & 1 \\
	-1 & 0 & 1
	\end{bmatrix} \\
P_{V} &= 
	\begin{bmatrix}
	-1 \\
	0 \\
	1 
	\end{bmatrix}
	*
	\begin{bmatrix}
	1 & 1 & 1
	\end{bmatrix}
	= 
	\begin{bmatrix}
	-1 & -1 & -1 \\
	0 & 0 & 0 \\
	1 & 1 & 1
	\end{bmatrix}
\end{align*}
\caption{Masks of the Prewitt gradient operator.}
\label{fig:PrewittMasks}
\end{figure}

\subsection{Sobel Operator}

\paragraph*{}
The quality of noise-suppresion of the Prewitt operator may be improved if we replace the indiscriminate mean average with simple approximation of Gaussian smoothing, which puts emphasis on pixels nearer to the pixel at which the average is computed. Such operator was proposed by Sobel.

\begin{figure}[h!]
\begin{equation*}
S_{H} = 
	\begin{bmatrix}
	-1 & 0 & 1 \\
	-2 & 0 & 2 \\
	-1 & 0 & 1
	\end{bmatrix}\qquad
S_{V} = 
	\begin{bmatrix}
	-1 & -2 & -1 \\
	0 & 0 & 0 \\
	1 & 2 & 1
	\end{bmatrix}
\end{equation*}
\caption{Masks of the Sobel gradient operator.}
\label{fig:SobelMasks}
\end{figure}

\paragraph*{}
Nixon and Aguado provide\cite{NixonAguado08} an interesting description of derivation of similar masks for higher dimensions, where both smoothing and differencing components are obtained from Pascal's triangle, however such masks are rarely used in practice. Sobel masks of bigger dimensions approximate the Gaussian smoothing while not making use of the separability of the Gaussian kernel, which make them inferior to the actual Gaussian smoothing, as performed by Canny edge detection method.

\paragraph*{}
\reffig{SobelVsBare} compares the results of bare central difference operator with those obtained using the Sobel operator. The achieved noise suppression is not spectacular, but still noticeable.


\twoFigures
{img/2DEdgeDetection/tooth_pd_amplitude}
{img/2DEdgeDetection/tooth_sobel_aplitude}
{Comparison of bare central difference gradient (on the left) and Sobel operator gradient.}
{SobelVsBare}
{\basicWidth}


\subsubsection{Scaling}

\paragraph*{}
When we derived the masks of Prewitt and Sobel operators we neglected the scaling of the smoothing operators, i.e. we used smoothing masks such as $[1 \, 1 \, 1]$, rather than $\frac{1}{3} [1 \, 1 \, 1]$. The operators were defined in this way to allow for faster computation and are commonly used in this form.

\paragraph*{}
Whenever the smoothing operators are to be compared (or used interchangeably, e.g. in edge detection with fixed gradient magnitude threshold) it is important to remember to normalize their results, multiplying the outcome by the scaling factor of the smoothing operator omitted in the first place ($\frac{1}{3}$ for the Prewitt operator and $\frac{1}{4}$ for Sobel). Such scaling was applied to the results of the Sobel operator in \reffig{SobelVsBare}.

\begin{refImpl}
Gradient operators discussed in this section are implemented by \studio filters \filter{GradientImage\_Mask}{ImageLocalTransform} and \filter{GradientImage}{ImageLocalTransforms}.
\end{refImpl}
\section{Topology}

\paragraph*{}
We have already seen operations defined in the context of set nature of the region as well as operations concerned with its spatial properties. The last set of transformations that we will discuss is built upon topological concepts such as neighborhood, connectivity or boundaries.

\subsection{Connectivity}

\paragraph*{}
Pixel connectivity defines the conditions in which we say that two pixels are \textit{connected} and as such is a key concept in the context of topological transformations. There is a well known paradox inherently associated with the definition of binary shapes connectivity on square grids\footnote{On triangular grids as well, but not on hexagonal ones.}.

\paragraph*{}
To demonstrate the paradox let us consider a closed not self-intersecting curve. Jordan Curve Theorem (and our common sense) requires that such curve should divide the space in which it lies into exactly two parts - interior and exterior. The paradox of pixel connectivity lies in the fact that this requirement is not met under either of two reasonable definitions of pixel connectivity.

\paragraph*{}
One possible definition of connectivity is 4-connectivity in which a pixel is consider connected to the pixels it shares an edge with, as demonstrated in \reftab{FourConnectivity}. Under this definition the curve on the right splits the space into three, rather than two connected components.

\newarray\fourNeighborhood
\readarray{fourNeighborhood}{%
0 & 0 & 0 & 0 & 0 &%
0 & 0 & 1 & 0 & 0 &%
0 & 1 & 2 & 1 & 0 &%
0 & 0 & 1 & 0 & 0 &%
0 & 0 & 0 & 0 & 0}

\newarray\fourExample
\readarray{fourExample}{%
0 & 0 & 0 & 0 & 0 & 0 & 0 & 0 &%
0 & 1 & 1 & 1 & 1 & 0 & 0 & 0 &%
0 & 1 & 0 & 0 & 1 & 0 & 0 & 0 &%
0 & 1 & 0 & 0 & 1 & 1 & 1 & 0 &%
0 & 1 & 1 & 1 & 0 & 0 & 1 & 0 &%
0 & 0 & 0 & 1 & 0 & 0 & 1 & 0 &%
0 & 0 & 0 & 1 & 1 & 1 & 1 & 0 &%
0 & 0 & 0 & 0 & 0 & 0 & 0 & 0}

\begin{table}[h]
\centering
\tabular{c c}

\gridDimensions{5}{5}\begin{tikzpicture}[scale=0.40]\drawslabs{fourNeighborhood}\end{tikzpicture} &
\gridDimensions{8}{8}\begin{tikzpicture}[scale=0.25]\drawslabs{fourExample}\end{tikzpicture}

\endtabular
\caption{4-connectivity kernel and a curve demonstrating a violation of the Jordan curve property.}
\label{tab:FourConnectivity}
\end{table}

\paragraph*{}
Another option to consider is 8-connectivity in which a pixel is considered connected to the pixels it shares a corner with, as demonstrated in \reftab{EightConnectivity}. Unfortunately in this case it is possible to construct a closed curve that does not split the space at all, i.e. curve the interior of which remains connected to the exterior, as demonstrated on the right.

\newarray\eightNeighborhood
\readarray{eightNeighborhood}{%
0 & 0 & 0 & 0 & 0 &%
0 & 1 & 1 & 1 & 0 &%
0 & 1 & 2 & 1 & 0 &%
0 & 1 & 1 & 1 & 0 &%
0 & 0 & 0 & 0 & 0}

\newarray\eightExample
\readarray{eightExample}{%
0 & 0 & 0 & 0 & 0 & 0 & 0 & 0 &%
0 & 1 & 1 & 1 & 1 & 0 & 0 & 0 &%
0 & 1 & 0 & 0 & 1 & 0 & 0 & 0 &%
0 & 1 & 0 & 0 & 0 & 1 & 1 & 0 &%
0 & 1 & 1 & 0 & 0 & 0 & 1 & 0 &%
0 & 0 & 0 & 1 & 0 & 0 & 1 & 0 &%
0 & 0 & 0 & 1 & 1 & 1 & 1 & 0 &%
0 & 0 & 0 & 0 & 0 & 0 & 0 & 0}

\begin{table}[h]
\centering
\tabular{c c}

\gridDimensions{5}{5}\begin{tikzpicture}[scale=0.40]\drawslabs{eightNeighborhood}\end{tikzpicture} &
\gridDimensions{8}{8}\begin{tikzpicture}[scale=0.25]\drawslabs{eightExample}\end{tikzpicture}

\endtabular
\caption{8-connectivity kernel and a curve demonstrating a violation of the Jordan curve property.}
\label{tab:EightConnectivity}
\end{table}

\paragraph*{}
Rosenfeld proposed\cite{Rosenfeld70} to address this problem by using different connectivities for foreground and background, which yields two feasible configurations:
\begin{itemize}
	\item 4-connectivity for foreground, 8-connectivity for background
	\item 8-connectivity for foreground, 4-connectivity for background
\end{itemize}
From now on we will use the terms 4-connectivity and 8-connectivity in the context of foreground connectivity, quietly assuming that the other type of connectivity is used for background.

\subsection{Connected Components}

\paragraph*{}
The necessity of splitting a region into its connected components (also referred to as \textit{blobs}) occurs naturally whenever the image contains a number of objects and single thresholding is used to extract one region collectively representing all of them. We have already seen an example of such proceeding in \reffig{DilationAppliedToFuses}. 

\paragraph*{}
Let us begin with a remark on more general problem of computing the connected components of any graph. In such case there are two ways one can follow: we can either traverse the components of the graph one component at a time (using either breadth-first or depth-first search) or go through the edges of the graph maintaining and updating a Union-Find structure representing our knowledge of connected components in graph (i.e. determining all connected components at the same time).

\paragraph*{}
As we represent regions as lists of pixel runs, the problem boils down to identification of the connected components of a set of pixel runs. Both of the approaches described above can be adapted to do so. Careful implementation of the DFS/BFS based technique performs in $O(R)$ time complexity, while the Union-Find based solution works in $O(R\log^{*}R)$.

\paragraph*{}
In either case to achieve target complexity it is crucial to carefully go through the region point runs determining the neighbors of each run (either building graph to which we will apply DFS/BFS or updating the Union-Find structure). Splitting the list of point runs into separate lists for each row within the region dimensions allow to do that in $O(R)$ time.

\subsection{Region Holes}

\paragraph*{}
The intuitive concept of a region hole may be formalized as follows:

\begin{description}
	\item[Region hole] is a connected component of region complement that is not adjacent to the boundaries of the region dimensions.
\end{description}

\paragraph*{}
An algorithm for extraction of the region holes may be derived straight from this definition. The one detail we should remember about is to use the background connectivity when computing the connected components of the region complement, as discussed before. 

\begin{refImpl}
\studio filter \filter{SplitRegionIntoBlobs}{RegionGlobalTransforms} extracts an array of region connected components while \filter{RegionHoles}{RegionFeatures} returns an array of region holes computed as decribed above.
\end{refImpl}
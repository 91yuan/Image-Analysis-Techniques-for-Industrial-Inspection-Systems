\section{Features}

\paragraph*{}
Once we have acquired a region that accurately represents the object that we intend to analyze, we may proceed to extraction of the region features. Features of two-dimensional shapes (discreet regions as well as continuous polygons) may be organized into two groups: statistical and geometrical.

\paragraph*{}
\textbf{Statistical features} of a shape are built upon statistical concepts such as mean or variance and may be computed directly from the coordinates of region pixels or polygon points, disregarding any spatial relations between them. Statistical shape features include, among others, its area, mass center and orientation (i.e. direction of the principal axis of inertia).

\paragraph*{}
\textbf{Geometrical features} are defined in the context of spatial relations between pixels or points contained in the shape. Some of the features, such as circularity factor, are numeric properties, others, such as smallest bounding circle, take form of geometric primitives.

\paragraph*{}
Almost all of the shape features of both kind that we are going to cover are equally applicable to pixel-precise regions and subpixel-precise polygons, which we will discuss in detail in the \refchap{ContourAnalysis} chapter. To avoid duplication, in this section we will focus only on few \textbf{region-specific features} and region-specific details of shape feature extraction.

\subsection{Statistical Features}

\paragraph*{}
Statistical features of two-dimensional shapes may be conveniently generalized as so called moments. Each moment is a numeric shape feature that sums a simple function of pixel (or point) coordinates over every pixel (or point) contained in the shape.

\paragraph*{}
Formally, we distinguish two types of moments, raw and central, the latter of which considers the coordinate arguments of the function in relation to the average (denoted $\overline{x}$, $\overline{y}$) of the appropriate pixel coordinate, thus achieving translation-invariance. For a given region $R$, its raw and central moments, are defined, respectively, as:

\begin{align*}
	m_{p,q} &= \sum_{(x,y) \in R} x^p y^q \\
	c_{p,q} &= \sum_{(x,y) \in R} (x-\overline{x})^p (y-\overline{y})^q
\end{align*}

where each $(p,q)$ for natural $p \geq 0$, $q \geq 0$ defines different moment. For instance, $m_{0,0}$ equals $\sum_{(x,y) \in R} 1$ and therefore computes the area of the region. $m_{1,0}$ computes the sum of x-coordinates of region pixels and so on.

\paragraph*{}
Both raw and central moments may be normalized, i.e. divided by the area of $R$, to achieve scale-invariance:
\begin{align*}
	m'_{p,q} &= \frac{1}{a} m_{p,q} \\
	c'_{p,q} &= \frac{1}{a} c_{p,q}
\end{align*}
where $a$ denotes the area (number of pixels) of R. For instance, $m'_{1,0}$ computes the average of pixel x-coordinates, i.e. $\overline{x}$ which together with $m'_{0,1}$ form the mass center of $R$.

\paragraph*{}
As previously indicated, we will discuss applications of these statistics in due course. The region-specific aspect of moment extraction that should be stressed here is that the region moments can be computed directly from their definition, as the number of region pixels is finite; as opposed to infinite number of points contained in continuous polygon.

\paragraph*{}
Moreover, low-order moments may be calculated very efficiently due to the RLE region representation that allow us to process a whole run of pixels in constant time, achieving complexity $O(r)$, where $r$ denotes the number of region pixel runs.

\begin{refImpl}
The following \studio filters extract the statistical features of a region: 
\filter{RegionArea}{RegionFeatures}, 
\filter{RegionElongation}{RegionFeatures}, 
\filter{RegionMassCenter}{RegionFeatures}, 
\filter{RegionMoment}{RegionFeatures},
\filter{RegionOrientation}{RegionFeatures},
\filter{RegionEllipticAxes}{RegionFeatures}. 
\end{refImpl}

\subsection{Geometrical Features}

\subsubsection{Contours}

\paragraph*{}
The contour of a region is a sequence of points defining its boundary (or an array of such sequences in case of disconnected region), as demonstrated in \reffig{RegionContour}.

\oneFigure
{BlobAnalysis/img/region_contours}
{Part of the contour of an example region.}
{RegionContour}
{\basicWidth}

\paragraph*{}
Such points may be computed very efficiently in RLE region representation. Let us observe that a contour of a region is defined by the sequence of its one pixel long, directed, \textbf{vertical segments} - once we have the vertical segments, adding horizontal sections between them is straightforward. 

\paragraph*{}
Moreover, all such vertical segments are easy to obtain from the RLE representation - these are in fact two sides of every point run in the region. Extraction of the contour therefore boils down to assigning successor to each vertical segment, i.e. answering a question for each such segment: which vertical segment will be next on the contour? This can be done one row of runs at a time with a careful linear scan of the point runs in the neighboring rows.

\subsubsection{Calculating Region Features from Path Features}

\paragraph*{}
Contour extraction is a particularly useful feature, because it essentially converts pixel-precise region to subpixel-precise polygons; thus allowing convenient implementation of a number of region features that are more natural to define for polygons.

\twoFigures
{BlobAnalysis/img/ContourFeatures_mid}
{BlobAnalysis/img/ContourFeatures_res}
{Bounding circle of a region computed indirectly as a bounding circle of its contour path.}
{ContourFeatures}
{\basicWidth}

\subsubsection{Other Features}

\paragraph*{}
Other geometric features are not inherently related to pixel-precise regions, and either may be computed using contour extraction together with corresponding operator for sub-pixel precise shapes (e.g. bounding rectangle) or are built on top of such operator (e.g. rectangularity factor). We will discuss these features in the \refchap{ContourAnalysis} chapter.


\begin{table}[h!]
	\centering
	\tabular{p{0.3\linewidth} p{0.65\linewidth}}
	\textbf{Feature} & \textbf{Description} \\ \hline 
	Bounding circle & The smallest circle containing the entire region.\\
	Bounding rectangle & The smallest rectangle (of any orientation) containing the entire region.\\
	Circularity & Measure of similarity to a circle. \\
	Diameter & The longest segment between any two pixels of the region.\\
	Perimeter length & Length of the region contours.\\
	Rectangularity & Measure of similarity to rectangle.
	\endtabular
	\caption{Other geometrical properties of a region.}
	\label{tab:OtherGeometricalFeatures}
\end{table}

\begin{refImpl}
The following \studio filters extract the geometrical features of a region: 
\filter{RegionBoundingBox}{RegionFeatures}, 
\filter{RegionBoundingCircle}{RegionFeatures}, 
\filter{RegionBoundingRectangle}{RegionFeatures}, 
\filter{RegionCircularity}{RegionFeatures}, 
\filter{RegionContours}{RegionFeatures}, 
\filter{RegionConvexHull}{RegionGlobalTransforms}, 
\filter{RegionConvexity}{RegionFeatures}, 
\filter{RegionDiameter}{RegionFeatures}, 
\filter{RegionPerimeterLength}{RegionFeatures} and 
\filter{RegionProjection}{RegionFeatures} and 
\filter{RegionRectangularity}{RegionFeatures}.
\end{refImpl}
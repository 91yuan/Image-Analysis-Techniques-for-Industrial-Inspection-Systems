\section{Edge-Based Matching}

\paragraph*{}
While the properties of normalized cross-correlation and efficiency boost offered by the pyramid matching make decent implementations of brightness-based template matching suitable for a range of typical applications, the algorithm retains a few weaknesses such as sensitivity to non-linear illumination changes and low capability of matching partially visible (occluded) template occurrences.

\paragraph*{}
As we have already noted, image edges are usually well preserved under disturbances of the illumination. Moreover, edges of an object define precisely its shape and therefore are a suitable discriminant for the identification of possible matches. The idea of edge-based template matching has been subject to extensive research and numerous measures of edge-based similarity between two images have been proposed. 

\subsection{Symmetric Methods}

\paragraph*{}
Perhaps the most fundamental classification of the edge-based template matching methods would be based on the relation between the template image and search image processing being performed by the algorithm. The earliest edge-based template matching methods were built upon edge extraction performed in the same way on both images. 

\paragraph*{}
One of the earlies methods was proposed\cite{Borgefors88} by Borgefors, who suggested to extract the edge pixels in both images and evaluate the matches using the \textbf{mean squared distance} between each edge pixel of the template image and the nearest edge pixel in the search image. 

\paragraph*{}
Rucklidge proposed\cite{Rucklidge95} a similar measure based on the \textbf{Hausdorff distance} between the edges extracted in both images, in which the difference between two images equals the maximum of two distances:
\begin{itemize}
	\item Longest distance between a template image edge pixel and any of the search image edge pixels.
	\item Longest distance between a search image edge pixel and any of the template image edge pixels.
\end{itemize}

\paragraph*{}
The major weakness of these and similar methods lies in the very idea of symmetric edge extraction. While the accurate edge extraction in the template image should not pose a problem, the assumption that the method which worked for the template image will consistently work for each search image, possibly under variable lightning conditions, is by far overoptimistic.

\paragraph*{}
Moreover, both methods are inherently sensitive to occlusions and as such have few, if any, advantages over the normalized cross-correlation grayscale-based template matching.

\subsection{Asymmetric Methods}

\paragraph*{}
To overcome the limitations of the symmetric edge-matching we need to move towards asymmetric methods. Steger proposed\cite{Steger02} to perform full edge detection in the template image, precisely identifying its contours; while each search image would be subject only to simple gradient computation on its entire area. Each possible match is then evaluated based on the correspondence of the gradient direction between the template and search image, but only at the positions of template edge pixels.

\paragraph*{}
This idea is demonstrated in \reffig{AsymetricEdges}, where we use colors from the HSV circle to represent gradient directions.

\fourFigures
{TemplateMatching/img/AssymetricEdges_template}
{TemplateMatching/img/AssymetricEdges_search}
{TemplateMatching/img/AssymetricEdges_template_edges}
{TemplateMatching/img/AssymetricEdges_search_gradient}
{Asymmetric edge processing - template image (a) is subject to full edge detection after which gradient direction at its edge pixels is computed (c), while for the corresponding section of the search image (b) computation of the gradient direction is performed indiscriminately on its entire area (d).}
{AsymetricEdges}
{.3\textwidth}

\paragraph*{}
Such approach immediately eliminates the problem of accurate edge extraction in the search image, as no such extraction is performed. Moreover, it utilizes matching of the edge directions, not only their presence, which was earlier demonstrated\cite{Olson97} by Olson et al. as advantageous in eliminating false-positive results in case of Hausdorff distance metric.

\paragraph*{}
The specific method of calculating the score of a match proposed by Steger is the simple sum of cosines of the angles between the corresponding gradient vectors, resembling the idea behind the normalized cross-correlation measure.

\paragraph*{}
Matching based on gradient direction does not preclude the application of pyramid matching schema. It is interesting to note that the asymmetric method actually outperforms the traditional grayscale-based template matching in terms of speed, as the evaluation of each possible match is limited to edge pixels of the template image; while the additional burden of computing the gradient of search image pyramid is not significant (compared with the dominating cost of the match evaluation).

\paragraph*{}
The method also works well against occlusions, as each missing edge pixel in the search image contribute at most the penalty of $-1$ to the final score, which equals in magnitude the input of a perfectly matched gradient direction, yielding cosine value of $1$. A detailed evaluation of the performance of this method is given in \cite{UlrichSteger02}.

\begin{refImpl}
Edge-based template matching is implemented in two \studio filters: \filter{CreateEdgeModel}{TemplateMatching} and \filter{MatchEdgeyModel}{TemplateMatching}.
\end{refImpl}